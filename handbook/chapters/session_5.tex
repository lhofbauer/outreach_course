\setchapterpreamble[u]{\margintoc}
\chapter{Model-based Analyses}
\labch{ana}

After completing this session successfully, you should be able

\begin{itemize}
    \item to present a power system analysis to an audience,
    \item to recognize about the limitations of energy models, and
    \item to structure a model-based energy analysis.
\end{itemize}

\section{A first power system scenario}
\begin{kaobox}[frametitle=Task]
Present your power system scenario in 1-2 minutes to your course leader and fellow students! Incorporate all the aspects mentioned in the homework task of session 4.
\end{kaobox}


\section{Limitations of model-based analysis}
\labsec{lim}

\paragraph*{Limitations of energy modelling studies}~\\

Power sector models and other energy models can be very important tools that provide quantitative, i.e., numerically-oriented, insights for energy planning activities of decision-makers. But model-based studies also have their limitations which are important to acknowledge. Limitations depend on the model and the purpose it is used for but a few general points to consider are
\begin{itemize}
\item the aspects of the real system that are not captured in the model,
\item the boundaries of the system the model captures, e.g., geographic areas or sectors that are linked to the system but not directly represented in the model,
\item elements of the system that are only captured in a simplified manner in the model, and
\item uncertainties when building models that consider the future.

\end{itemize}

It is important to consider and acknowledge these limitations to be able to provide a justifiable and well-substantiated analysis.


\begin{kaobox}[frametitle=Task]
Consider again the power sector model introduced in the previous session and used for the very short analysis for your homework. What limitations does the model and analysis have? Go through the list above with your neighbour and note down all limitations you can think of.
\end{kaobox}

\section{Analysing and discussing results}
\labsec{res}
When analysing model results and drawing and communicating conclusions based on them, it is important to take into account the structure and data inputs of the model, as well as related limitations. As discussed in previous sessions, models are often, in particular when they consider complex energy systems, a stark simplification of reality, designed for a particular purpose, e.g., to address a particular set of questions. Hence, it is crucial to reflect about how to interpret model results and what conclusions can be drawn and which not.


\begin{kaobox}[frametitle=Task]
Imagine a study looking at a power system for Greater London is performed using the simple power sector model spreadsheet introduced in the last session. The analysis introduces two different scenarios for the year 2035. Which of the following statements/conclusions could be realistically drawn from the model results, which rather not? If not, why?

\begin{itemize}
\item In 2035, there will be 100 MW of PV panels installed on rooftops in London.
\item In scenario A, the installed capacity of PV panels reaches 100 MW in 2035.

\item The results indicate that solar panels can play a larger role than wind power in supplying power to London.
\item The results show wind turbines are cheaper than solar panels.

\item To implement the power system for scenario A the Greater London Authority will have to spent 2 billion pounds.
\item As compared to scenario A, scenario B requires only half of the investment in power generation in London.

\item The analysis shows that the Mayor of London and other decision-makers should introduce measures to foster the uptake of solar PV.
\end{itemize}

\end{kaobox}




\section{Final course work}

\textbf{Investigate a future renewable power system in your local area by exploring two scenarios!}


\textcolor{red}{[To start with, you only need to prepare a draft of this assignment. For the draft, try to implement both of your scenarios using the model and write down bullet points for each of the sections of your report. Please submit the draft assignment before the deadline set by your course leader.]}

Your course leader will provide you with a basic power sector model for the local area you have chosen in session 4. The model will be based on a spreadsheet with the same structure as the models used during previous sessions. The model includes all necessary values and the capacity factors for wind and solar power generation specific to your chosen area. Using your knowledge from the previous sessions, apply this model to analyse two different scenarios for a future power system which only uses local renewable power resources to meet the electricity demand of your area.

To create different scenarios from the base model, you can do one or more of the following:

\begin{itemize}
\item use different capacities of installed solar PV, wind turbines, etc.
\item use different technology cost by researching estimates for the future cost of technologies
\item change the demand
\item ...
\end{itemize}


Your submission is a 2000 words long report that describes your scenario analysis for local decision-makers. Your report should include the following sections:

\begin{enumerate}
\item Introduction: You can write about your motivation to perform this analysis (Why are you planning a renewable power system?), the background, for example about the efforts your local area is already pursuing (for example, do they already have a climate target?), and the aim of your study (Who will your analysis help?). In the end of the introduction, you can also explain the structure of the rest of the report (`The report is structured as follows. In the next section [...]').
\item Methodology: In this section, you should explain the model that you are using. You can talk about the assumptions (What demand do you use? What cost data?) and about how the model works, i.e., how it considers the balance between demand and supply of electricity (What is the temporal resolution? ... ).
\item Scenarios: You can split this section into two, one for each scenario. Explain the background of your scenarios by giving a narrative/storyline (What general, qualitative assumptions do you make, e.g., is it a scenario that assumes ambitious action?). Then discuss the power system based on your modelling results (How much solar and wind power will be installed? What are the cost? What are the advantages and benefits? What are the drawbacks? ... ). Use references to other sources to support your arguments. Make sure to include the graph from the spreadsheet to present your results. Reading through your scenario description, the reader should have a good idea about this `future' you are exploring.
\item Discussion: After explaining the scenarios in the previous section, you can discuss the results here (What do the results mean for local decision-makers? How do both scenario compare? ). It would be great to put the analysis in the context of other work (you can potentially use your research for homework 3). Make sure to use references to point to the others' work you are referring to. Also, please discuss the limitations of your model and analysis (What are the simplifications of the model? What other aspects does your analysis not capture?).
\item Conclusions: Here, you should give a short summary of your analysis and some concluding remarks (What is the core message of your work to local decision-makers? ...).
\end{enumerate}


Please have the following points in mind when working on your assignment:

\begin{itemize}
\item Your assignment needs to be 2000 $\pm$ 10\% words long (the list of references and captions of figures do not count towards the word limit).
\item Make sure to use references when you refer to other people's work.
\item You need to submit your report on the learning platform introduced by your course leader. Plan to do this well before the deadline to avoid technical problems at the last minute.
\end{itemize}

